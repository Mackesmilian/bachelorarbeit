\externaldocument{main.tex}
As with any framework, installation comes first.
Electron can be simply installed using npm, the node package manager: \parencite{npm}
\begin{lstlisting}[caption=Installing Electron via npm]
    npm install -g electron
\end{lstlisting}
This installs Electron as a global dependency. 
The next step is to create a project directory and a package.json file for the Electron configuration, among other things. 
The entry point for any Electron application is a JavaScript file. 
In this specific case it is named app.js:
\begin{lstlisting}[caption=Minimal package.json for an Electron application]
{
    "name": "pze",
    "version": "1.0.0",
    "main": "app.js"
}
\end{lstlisting}
Note that the name property on line two is an abbreviation of the German term for this application:
Projektzeiterfassung, meaning project time keeping and abbreviated to PZE.\paragraph{}
As mentioned the starting point for any Electron application is a JavaScript file. 
This file (in this case app.js) is responsible for loading all required parts of the Electron application 
and any windows that it should display.
See the following example for creating a browser window in electron:
\begin{lstlisting}[caption=Creating a window in Electron on ready event]
const { app, BrowserWindow, ipcMain } = require("electron");
let mainWindow;
function createWindow() {
  mainWindow = new BrowserWindow({
    width: 800,
    height: 600,
  });
  mainWindow.loadURL(
    url.format({
      pathname: path.join(__dirname, `/dist/pze/index.html`),
      protocol: "file:",
      slashes: true,
    })
  );
  mainWindow.on("closed", function () {
    mainWindow = null;
  });
}

app.on("ready", createWindow);
\end{lstlisting}
After importing the necessary parts of Electron one needs to create a window. 
As the "ready" event is fired once the app has finished loading one can listen to said event
and then call the createWindow() function to create a window and more importantly load a file to display.
On line three a window is created with the specified dimensions, 800 by 600 pixels in this case. 
Electron's approach to window sizing means that a developer can specify different window sizes for each
window directly in code rather than having to specify such properties in a configuration file, as is 
the case with NW.js. \parencite{jensen2017}
Furthermore, parameters for the initial positioning of the window can also be passed to the BrowserWindow constructor.\paragraph{}
The next step is to load a URL.
This URL needs to point to an HTML file containing the content to display. 
In this case the output file of the Angular part of this application is loaded. 
Finally, after the window has been closed the variable will be set to null.\paragraph{}
However, not every platform behaves the same when an application is closed by clicking the X symbol in the top 
right or left corners. 
As opposed to Windows and Linux, MacOS does not quit an application once a window is closed. 
To check for the platform so the correct behaviour can be implemented one can use the process.platform \parencite{nodeDocs} property
provided by node.js:
\begin{lstlisting}[caption=Quitting and terminating an Electron application]
app.on("window-all-closed", function () {
    if (process.platform !== "darwin") app.quit();
});
\end{lstlisting}
As shown the code listens for the "window-all-closed" event and if the user's operating system is anything but MacOS 
the app quits. 
These steps form the basic scaffolding of an Electron project and can now be run with the electron command.