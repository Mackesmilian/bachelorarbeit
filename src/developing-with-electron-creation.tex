As with any framework, installation comes first.
Electron can be simply installed using npm, the node package manager: \parencite{npm}
\begin{lstlisting}
    npm install -g electron
\end{lstlisting}
This installs Electron as a global dependency. 
The next step is to create a project directory and a package.json file for the Electron configuration, among other things. 
The entry point for any Electron application is a JavaScript file. 
In this specific case it is named app.js:
\begin{lstlisting}
{
    "name": "pze",
    "version": "1.0.0",
    "main": "app.js"
}
\end{lstlisting}
Note that the name property on line two is an abbreviation of the German term for this application:
Projektzeiterfassung, meaning project time keeping and abbreviated to PZE.
