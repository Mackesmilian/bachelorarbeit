\externaldocument{main.tex} 
For communicating between the front- and back-end Electron provides the ipcRenderer and ipcMain modules.
In the front-end part events are emitted with ipcRenderer. 
Electron offers various different ways of sending and receiving events.
The methods to listen to events on the ipcRenderer are as follows: \parencite{electronDocs}
\begin{itemize}
    \item ipcRenderer.on(channel, listener):\\
    This method listens to a channel and when a message on the corresponding 
    channel arrives, the listener is called.
    \item ipcRenderer.once(channel, listener):\\
    Acts similarly to ipcRenderer.on, but the listener is removed after the invocation
\end{itemize}
With ipcRenderer.removeListener(channel, listener) or\\
 ipcRenderer.removeAllListeners(channel) listeners can be removed.
Note that with ipcRenderer.removeAllListeners(channel) the channel parameter is optional and when omitted
all listeners get removed.
The ipcMain Event Emitter has the same methods as ipcRenderer to 
listen to events, with the only addition being ipcMain.handle(channel, listener).\paragraph{}
Similarly to receiving events, Electron offers multiple methods to send events from the ipcRenderer Event Emitter. 
Those methods include: \parencite{electronDocs}
\begin{itemize}
    \item ipcRenderer.send(channel, ...args):\\
     This methods sends an asynchronous message to ipcMain via the specified channel. 
    \item ipcRenderer.invoke(channel, ...args):\\
    This methods also sends an asynchronous message. 
    It does however expect a result and returns a Promise<any> to resolve the response.
    Note that on ipcMain handle() should be used to intercept these events.
    \item ipcRenderer.sendSync(channel, ...args):\\
     Same as ipcRenderer.send(channel, ...args) with the difference of 
    expecting a synchronous result.
    \item ipcRenderer.sendTo(webContentsId, channel, ...args):\\
    A method for sending an event directly to a specified window 
    using the webContentsId parameter.
    \item ipcRenderer.sendToHost(channel, ...args):\\
     Behaves the same as ipcRenderer.send(channel, ...args), with the 
    difference being that the event is sent to the <webview> element rather than the main process.
\end{itemize}
Having listed the options provided by Electron for inter-process communication the next step is to integrate the ipcRenderer into Angular.
For this, there are multiple solutions. 
To simplify development, developers can use an NPM module called ngx-electron developed by Thorsten Hans. \parencite{ngxElectron}
Ngx-electron makes calling Electron's API from Angular simpler for developers by exposing all methods available within Electron's
render process.
This means the above mentioned methods would all be accessible through this module. 
To include it in the Angular application one needs to import the module:
\begin{lstlisting}[caption=Importing ngx-electron's ElectronService into Angular]
import { NgModule } from '@angular/core';
import { BrowserModule } from '@angular/platform-browser';
import { HttpClientModule } from '@angular/common/http';

import { NgxElectronModule } from 'ngx-electron';
@NgModule({
  declarations: [
    AppComponent,
  ],
  imports: [
    BrowserModule,
    BrowserAnimationsModule,
    MaterialModule,
    HttpClientModule,
    NgxElectronModule,
  ],
  bootstrap: [AppComponent],
})
export class AppModule {}
\end{lstlisting}
After then importing the ElectronService class from the ngx-electron module in the necessary components 
one can send or listen to events by accessing the API:
\begin{lstlisting}[caption=Sending an event with ngx-electron]
this.electronService.ipcRenderer.send('getProjects');
\end{lstlisting}
For this solution to work there is a caveat however: 
If one tries to run this example with the latest stable Electron release (at the time of writing 17.1.0), the 
following error will be encountered:
\begin{lstlisting}[caption=Error with ngx-electron and Electron > 13.6.9]
Error: node_modules/ngx-electron/lib/electron.service.d.ts:17:31 - error TS2694: 
Namespace 'Electron.CrossProcessExports' has no exported member 'Remote'.

17     readonly remote: Electron.Remote;
                                 ~~~~~~
\end{lstlisting}
This error occurs because in version 12 of Electron, the remote module has been deprecated and in version 14
removed from Electron itself and moved to another package, @electron/remote. \parencite{electron14Blog}
This of course leads to an error because ngx-electron cannot locate the required module. 
Because the latest release of ngx-electron (version 2.1.1) happened in October of 2019, one can 
assume that this issue (which is still marked as open on GitHub as of March 2022) will not be fixed in 
the foreseeable future. \parencite{namespaceError}
What this means for developers is that if they wish to use ngx-electron, the latest usable stable release
of Electron is 13.6.9. 
While said release was last updated (at the time of writing) on February 2nd 2022, being forced to use
a deprecated feature and being locked into a specific version of any framework can pose a worry to 
many developers. \paragraph{}
As an alternative, developers can skip the use of ngx-electron and directly import the required electron module.
\begin{lstlisting}[caption=Requiring Electron in Angular]
const electron = (<any>window).require('electron');
\end{lstlisting}
Sending an event through the ipcRenderer would be very similar to the code sample above which is using 
ngx-electron:
\begin{lstlisting}[caption=Sending an event through ipcRenderer]
// Using ElectronService from ngx-electron
this.electronService.ipcRenderer.send('getProjects');

// Directly accessing ipcRenderer
electron.ipcRenderer.send('getProjects');
\end{lstlisting}
The advantage would be greater future-proofing because the use of a newer Electron stable release would
be possible.
Ultimately which solution is chosen is at the developer's discretion. 
For this proof-of-concept the ngx-electron module will be used.\paragraph{}
In figure six the flow of data is pictured when the application communicates with
the back-end.
One main goal of this application is to take advantage of Electron's arguably biggest asset over 
traditional web applications: offline functionality.
Due to the Electron back-end being developed with Node.js one can access all features available to Node.js 
which in a traditional application would not be accessible on the client's machine.
This means one can use the fact that Electron allows to read and write to the User's file system. 
It is therefore very trivial to implement storage of data on each client's machine.  
See the following sequence diagram for an overview of how data to back up is fetched from the API 
on startup.
\begin{figure}[H]
  \centering
  \label{fig:pze-sequence-diagram-electron}
  \includegraphics[width=1\textwidth]{pze-sequence-diagram-electron}
  \caption{Fetch Backup Data Sequence Diagram:
  A sequence diagram illustrating how data is fetched on startup}
\end{figure}
As pictured the API is queried if the application is online.
This is checked via the window.navigator.onLine property.
This property returns whether the client is connected to a network, be it LAN or WLAN.
The issue with this is however, that by querying whether the client is connected to a network,
the client could be connected to a network while being unable to access the internet or a specific
required resource.
Another important point about this property is the fact that browsers handle it differently.
Firefox and Internet Explorer return a false value if the browser is switched to offline or the browser
cannot establish a network connection whereas in Chrome and Safari the only case where false is returned
is when the browser cannot connect to a network. \parencite{onLineMdn}
This is to highlight another advantage of Electron: Since the browser is based on Chromium, 
developers do not need to take other browser implementations into consideration.\paragraph{}
However, as mentioned navigator.onLine is not reliable enough, because it cannot make a distinction
between no network availability or having no internet connection, let alone checking the 
host directly.
To work around this, one can send a HTTP GET request directly to the server. 
For this a /test endpoint was implemented with a 204 no content response. 
It is then trivial to send a get request and check for a 204 status.\paragraph{}
After checking whether the API is reachable, GET requests are sent to the API for the top 100 entries
and all records of project, customer and template entities. 
The returned data is then stored in corresponding JSON files as arrays. 
Note that because this is a proof-of-concept, no encryption and authentication is performed.
In a productive environment steps should be taken to only transmit data the user is authorised 
to access and ideally, stored data should be encrypted as it is stored on a client machine.\paragraph{}
After fetching data from the API to store locally the program then posts new data stored on the 
client machine to the API:
\begin{figure}[H]
  \centering
  \label{fig:pze-sequence-diagram-electron-saveNew}
  \includegraphics[width=1\textwidth]{pze-sequence-diagram-electron-saveNew}
  \caption{Post new Data to API:
  A sequence diagram illustrating how new data stored on the client machine is posted to the API}
\end{figure}
After the process outlined in figure seven, the file system is checked for new data. 
If said data exists it is then sent to the API via HTTP POST and persisted to the database.
Once this has finished the local copy of the now persisted data is deleted to avoid duplicates.\paragraph{}
Having outlined how the startup procedure for this application looks like, there is still the question of 
how one can send events and pass data between Angular and Electron.
As explained at the beginning of this chapter, ngx-electron's ElectronService will be used to send events 
to Electron from Angular. 
The process of fetching data from the Electron back-end is very similar between each entity so 
in order to illustrate the workflow see the following example of fetching customers.
\begin{lstlisting}[caption=Sending an event using ElectronService]
  this.electronService.ipcRenderer.send('getCustomers');
\end{lstlisting}
The code above emits an event called getCustomers. 
This event will be sent through the ipcRenderer EventEmitter and Electron will now listen 
for it on ipcMain.
\begin{lstlisting}[caption=Listening for an event]
  ipcMain.on("getCustomers", (event) => {
});
\end{lstlisting}
This listener will execute the callback function once it intercepts the getCustomers event. 
Within the callback function the data is read from the file system:
\begin{lstlisting}[caption=Reading data in event callback function]
  ipcMain.on("getCustomers", (event) => {
    var pathName = path.join(__dirname, `/files/pze/customers.json`);
    fs.readFile(pathName, (error, data) => {
      if (error) {
        console.error(err);
        event.reply("getCustomersReponse", null);
      } else {
        let customers = JSON.parse(data);
        let newCustomers = getNewObjects("customers_new.json");
        newCustomers.forEach(function (newCustomer) {
        customers.push(newCustomer);
      });
      console.log(customers);
      event.reply("getCustomersReponse", customers);
    }
  });
});
\end{lstlisting}
Note the use of event.reply(). 
There are a couple advantages to using event.reply(). 
Firstly, it is asynchronous. 
This means it won't block any other code from executing and one can therefore expect better performance 
than when replying with a synchronous message.
Another advantage is the fact that event.reply() automatically handles messages that originate from
different frames to the main frame. 
That is to say should the original event be fired from an iframe, this helper method will 
handle such a reply in the correct manner. \parencite{electronDocs}\paragraph{}
Currently, the application can send an event to Electron and Electron will respond with 
the requested data. 
However as of now there is no way to intercept the response and therefore data.
To achieve this there are multiple different approaches to arrive at a solution. 
One solution is to adhere to the observer pattern, a software design pattern which is used 
for event handling and asynchronous programming while handling multiple values.
In this pattern an object maintains a list of observers which are then notified by the 
subject when an event (or rather state change occurs).
Observables are widely used in Angular for the above reasons. \parencite{angularDocsObs}\paragraph{}
In order to create such an Observable which notifies its subscribers when a change occurs (in this 
case when an event from Electron is emitted), one should use the Observable<T> constructor provided
by Angular.
\begin{lstlisting}[caption=Creating an Observable for ipcRenderer]
fetchAllCustomers(): Observable<Customer[]> {
  return new Observable<Customer[]>((observer) => {
    this.electronService.ipcRenderer.on('getCustomersReponse', (event, arg) => {
      this.ngZone.run(() => {
        observer.next(arg);
      });
    });
    electron.ipcRenderer.send('getCustomers');
  });
}
\end{lstlisting}
The first step is to create an Observable, with the type parameter being an array of customers.
Following that is the call to ipcRenderer via ElectronService where a listener is created for 
the getCustomersReponse event which is the reply sent in the Electron back-end.
At the end the actual trigger event for fetching the data is sent. \paragraph{}
One important consideration is the use of NgZone. 
Zones in this context refer to persisting execution context across different asynchronous tasks.
These zones can be compared to the so-called \acrfull{tls}.
\acrshort{tls} is a method of assigning locations for data specific to a thread. \parencite{microsoftTls}
This means data is not stored globally but privately available for each thread.
Analogous to \acrshort{tls} Zones exist in JavaScript virtual machines to encapsulate 
change detection into separate execution contexts. \parencite{angularDocsZone}\paragraph{}
The underlying functionality is based on Zone.js which creates said execution contexts.
These contexts can persist asynchronous operations and Zone.js also provides lifecycle hooks 
for these asynchronous operations. \parencite{zoneJs}\paragraph{}
Angular builds on top of Zone.js with the provided NgZone service.
This service creates a zone named angular which then triggers change events when a
synchronous or asynchronous method is executed and when there is no scheduled microTask.
This separation of contexts into zones creates a problem for detecting event changes which 
happen outside of the angular zone, such is the case with Electron.
To overcome this, NgZone service offers the run() method. 
By calling this run method the code is brought back into Angular's zone and executed there 
which means change detection is carried out on all operations inside that run function. \parencite{angularDocsZone}\paragraph{}
These are the necessary steps to communicate between Angular and Electron. 
With this implementation it is trivial to pass data to Electron and return data if necessary. 
Having implemented all necessary functionality it is now time to package and deploy the Electron application.
