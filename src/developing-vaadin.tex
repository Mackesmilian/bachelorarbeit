\externaldocument{main.tex}
Having outlined what steps are needed to create, implement, and deploy and Electron application
with Angular a look at how the same thing can be achieved with Vaadin will be taken.
As Vaadin is a Java Framework it is compatible and recommended to use with Spring. \parencite{vaadinDocs}
The initial step would be to include a dependency to Vaadin with a build tool of choice. 
This case study will use Vaadin 14 a proven \acrfull{lts} release with the newest being Vaadin 23 \acrshort{lts}.\paragraph{}
To create an application layout in Vaadin a parent layout class is required:
\begin{lstlisting}[caption=Creating a MainLayout.java class]
@Theme(value = Lumo.class, variant = Lumo.LIGHT)
public class MainLayout extends AppLayout {
    
    public MainLayout() {
        DrawerToggle toggle = new DrawerToggle();
    
        H1 title = new H1("PZE");
        title.getStyle()
                    .set("font-size", "var(--lumo-font-size-l)")
                    .set("margin", "0");
    
        addToDrawer(getTabs());
        addToNavbar(toggle, title);
    }
    private Tabs getTabs() {
    .
    .
    .
    }
    
    private Tab createTab(String viewName,Class<? extends Component> navigationTarget) {
    .
    .
    .
    }
    
    }
\end{lstlisting}
For sake of brevity the details of \lstinline[columns=fixed]{getTabs()} and \lstinline[columns=fixed]{createTab()} 
have been omitted. 
The class annotation \lstinline[columns=fixed]{@Theme} sets the application theme with Lumo being a Vaadin provided theme.
The \lstinline[columns=fixed]{MainLayout} class extends \lstinline[columns=fixed]{AppLayout} which is a Vaadin provided
layout and serves the purpose of having a view with a drawer for navigation on the left of the screen.  
Said drawer is created in the constructor with each tab representing a view.\paragraph{}
In order to create a Vaadin view one must create a view class. 
\begin{lstlisting}[caption=Creating a Vaadin view]
@Route(value="customer", layout = MainLayout.class)
public class CustomerView extends VerticalLayout {
    
    private Grid<Customer> grid = new Grid<>();
    private CustomerRepository repository;
    
    public CustomerView(@Autowired CustomerRepository repository) {
        this.repository = repository;
        setSizeFull();
        createGrid();            
        add(grid);
    }
    
    private void createGrid() {
        grid.setSizeFull();
        grid.addColumn(Customer::get_id).setHeader("Name");
        grid.addColumn(Customer::getAddress).setHeader("Address");
        grid.setItems(repository.findAll());
    }
}
\end{lstlisting}
This represents a basic implementation of a working view with a grid showing instances of an entity, in this case Customer.
After defining the view rout in the class annotation as well as setting the parent layout to the previously defined 
MainLayout the View class extends VerticalLayout which is as the name implies a Layout with vertical ordering. 
Other types of layouts are HorizontalLayout, Div, and SplitLayout.\parencite{vaadinDocs}\paragraph{}
The first field of the view is a Grid object which again is a Vaadin provided component. 
This grid is bound to a type and as seen in \lstinline[columns=fixed]{createGrid()} method columns can be added with
a getter for each property of the object. 
Loading the items to display is as trivial as calling the \lstinline[columns=fixed]{grid.setItems()} method which
accepts a collection of the previously specified class. 