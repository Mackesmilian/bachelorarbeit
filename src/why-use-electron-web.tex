\externaldocument{main.tex}
In contrast to desktop applications, the relatively low barrier for entry in web development thanks to the ease of learning
the basics of HTML, CSS and JavaScript makes it much easier for developers to create complex web applications. 
With the amount of open source frameworks developers of web apps have a large selection of different solutions to fit
their specific use case.
Also, package managers like npm offer a large selection of readily available, well established packages for developers to use
and enhance their projects with.\paragraph{}
Another big advantage of web applications is that they are platform-independent. 
A web application can be reached on any reasonably modern device which runs a web browser.
There is no need to create a separate version for all the operating system one wants to support and 
websites can also easily be accessed on mobile devices.\paragraph{}
As described in the previous chapter \ref{subsubsec:desktop-applications}, web applications need continuously running infrastructure such as
web servers and databases. 
While this constitutes a disadvantage, it also comes with a big benefit for developers, as they can strictly control which version 
a client uses.
Furthermore, the access to real-world data in said databases makes reproducing bugs much simpler. \parencite{jacobs2005}