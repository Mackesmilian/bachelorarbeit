\externaldocument{main.tex} 
After taking the steps to create and scaffold the Angular application, some configuration work needs to be done.
In the angular.json file the output directory needs to be set to the previously specified directory where 
Electron looks for the index.html file:
\begin{lstlisting}[caption=Angular configuration for Electron]
"options": {
    "outputPath": "dist/pze",
},
\end{lstlisting}
An index.html file is required regardeless of the approach to front-end development.
Another modification is required for the start scripts in the package.json file where the angular application has to be built 
and then the electron application:
\begin{lstlisting}[caption=Start scripts for Electron and Angular]
"scripts": {
        "start": "ng build --base-href ./ && electron .",
},
\end{lstlisting}
With this npm start can be executed and first ng build will be called, building the angular application after which the 
Electron start script will be called and the Electron application will be started.\paragraph{}
Now the app is up and running. 
An Electron instance will start with the Angular application running inside it. 
To speed up the development process, the Angular front-end should be developed separately. 
Not only does this lead to looser integration of Angular and Electron but when saving edits made to Angular 
developers need to restart the Electron build process which takes considerably longer than just reloading 
the Angular app. 
This means during development - and as long as the Electron part is not needed - developers should start the 
application with ng serve as changes will be adopted almost instantly whereas with Electron one needs to run the 
entire start script again.\paragraph{}