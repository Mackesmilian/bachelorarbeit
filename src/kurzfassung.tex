Web Applikationen haben über die letzten Jahre einen Zuwachs 
in Popularität und Komplexität erfahren was zu einem regelrechten
Ruck von Desktopapplikationen zu Web Applikationen geführt hat. 
Mehr Internetnutzer*innen, schnellere Internetgeschwindigkeiten 
und finanzielle Gründe haben zu diesem Wechsel beigesteuert. 
Dadurch stellt sich natürlich die Frage ob dieser Weg der 
einzige ist um Applikationen an Nutzer*innen zu bringen. 
Schließlich setzen Web Applikationen beispielsweise eine stabile 
Internetverbindung vor.
Um diese Probleme zu umgehen sind einige Framworks entstanden, die
versprechen, solche Nachteile von Web Applikationen zu beseitigen. 
Eines solcher Framworks ist Electron, welches Entwickler*innen
ermöglicht, native Desktopapplikationen ausschließlich mit 
Webtechnologien wie JavaScript, HTML und CSS zu entwickeln.
Das Ziel dieser Arbeit ist eine Evaluierung dieses Framworks
mit Blick auf Vor- und Nachteile welche von Entwickler*innen
in Betracht genommen werden könen, wenn Electron zum Einsatz kommen würde. 
Um diese Lücke zu schließen wurde eine Fallstudie durchgeführt 
die den Entwicklungsprozess mit Electron untersucht.
Daraus resultierend wurden Vor- und Nachteile identifiziert 
sowie Empfehlungen für Organsationen und Entwickler*innen formuliert, 
worauf bem Einsatz von Electron geachtet werden kann. 
Für künftige Forschungsarbeiten bezüglich Electron - oder auch anderen
vergleichbaren Frameworks - werden weitere relevante Problemstellungen
identifiziert wie zum Beispiel die Frage inwiefern sich Electron in 
einen produktiven Entwicklungsprozess einbinden lässt und welche 
Überlegungen in Bezug auf Wartung, Versionierung und Weiterentwicklung
in einem unternehmerischen Kontext relevant sind.