\externaldocument{main.tex}
Having described one of the advantages of Electron over web applications, namely 
extensive offline functionality, there are many more aspects of desktop applications
which one can access through Electron.
In this section it will be described how one can create context menus and take advantage 
of key bindings through Electron.\paragraph{}
Without specifying any menu items Electron displays basic menu options which are familiar from 
web browsers such as an edit menu with "undo" and "redo" options or a view menu in which developer
tools can be toggled among other things.
In order to create custom menu items, one must define a template in JSON form:
\begin{lstlisting}[caption=Menu template for Electron.]
const menuTemplate = [
    {
        label: "File",
        submenu: [
        {
            label: "Fetch Data",
            click: checkIfOnlineAndFetch,
        },
        {
            label: "Synchronise new Data",
            click: checkIfOnlineAndSync,
        },
        {
            label: "Toggle Offline Mode",
            click: enableOfflineMode,
        },
      ],
    },
];
\end{lstlisting}
This example creates a menu item with the label "File" and three sub menus, each of which are given 
a click handler. 
To enable this menu template Electron's Menu class offer a method named buildFromTemplate():
\begin{lstlisting}[caption=Enabling the custom menu.]
const Menu = require("electron");
const menu = Menu.buildFromTemplate(menuTemplate);
Menu.setApplicationMenu(menu);
\end{lstlisting}
The above code creates a menu from the previously defined template and then sets the now created menu
as the current application menu. 
In this case three options were added, first the option for users to manually fetch data from the API,
then the option to manually synchronise newly created, locally saved objects and finally the option
to toggle offline mode, should users wish to do so.\paragraph{}
Another useful Electron feature is the fact that key bindings can be freely chosen. 
Of course, one can also implement key bindings and combinations with regular web applications.
However, one needs to take two factors into consideration:
Key press events are handled differently across browsers and certain combinations can only be 
overridden with workarounds and arguably should not be implemented. \parencite{mdnKeyboardEvent}
For example, the \lstinline[columns=fixed]{control + N} combination is universally understood to create 
a new of something, whatever the application in question implements.
With web application, one is constrained by already existing browser key combinations, such as \lstinline[columns=fixed]{control + N} or 
\lstinline[columns=fixed]{control + T}.
For example in chrome, such an override is not possible:
\blockquote{In Chrome4, certain control key combinations [Cntr-N, Cntrl-W, Cntr-T] have been reserved for browser 
usage only and can no longer be intercepted by the client side JavaScript 
in the web page.}
\parencite{chromeIssueTrackerV4}
In addition to having to take different browsers and their versions into consideration one could also argue
that by overriding bindings such as \lstinline[columns=fixed]{control + N} one would disregard usability
conventions because users would most likely expect \lstinline[columns=fixed]{control + N} to open a new 
browser window and not something website-specific. \parencite{usabilityHeuristics}\paragraph*{}
As far as desktop applications go, however, one does not have to take such points into consideration.
In this example a listener on \lstinline[columns=fixed]{control + N} was added to open an edit dialog:
\begin{lstlisting}[caption=Adding a global keyboard shortcut.]
globalShortcut.register("CommandOrControl+N", () => {
    mainWindow.webContents.send("openDialog");
});
\end{lstlisting}
This creates a global shortcut for the \lstinline[columns=fixed]{control + N} or \lstinline[columns=fixed]{command + N} 
(for MacOs) combination. 
Once it is registered an event is emitted which can be intercepted in the front-end. 
Another way of adding keyboard shortcuts is to register them locally. 
That is to say these shortcuts will only trigger when the application is focused. \parencite{electronKeyboardShortcuts}
Furthermore, keyboard shortcuts can be added as so-called accelerators which means the combination will
be displayed alongside the respective option in the application menu. \paragraph{}
These are just some points Electron provides in terms of native Desktop APIs. 
Developers can leverage much more features such as notifications, Drag \& Drop, and device access among other things.
Additionally, some platform-specific APIs such as Linux launcher actions or recent documents for Windows and MacOs
exists as well.