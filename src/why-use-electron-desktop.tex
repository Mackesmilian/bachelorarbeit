Desktop applications used to be the standard way of delivering software to the end user.
Users had to go to a store, buy a CD-ROM, check the system requirements and install the software 
on their machine.
This does of course come with a number of benefits over web applications.\paragraph{}
One advantage is that desktop applications aren't reliant on an internet connection.
Web applications obviously fail here, as they are accessed over the internet.
Furthermore, this reliance on an internet connection leads to more issues when the application
is very feature-rich and/or has to support large files.
An image editing software as a web application for example can run into limitations when being
used with high-resolution images. 
Similarly, desktop applications start instantly without having to download resources over the 
internet. 
In such cases, desktop applications have an edge over comparable web applications. \parencite{jensen2017}\paragraph{}
There are of course also benefits to using desktop applications from a developer's perspective.
As a developer, one does not have to worry about users accessing their web applications over different web browsers 
as the choice of browser is at the user's discretion. 
This means not having to consider how different browsers interpret CSS, for example and always being sure about how the UI 
is being displayed and how the application's code is interpreted.
Another benefit is the fact of tighter integration with the user's OS.
Browser security limits the use of hardware and can lead to challenges for certain use cases. \parencite{jensen2017}\paragraph{}
Moreover, having an installable desktop application means not having to continuously support all the necessary infrastructure.
There simply is no need to run servers, databases and such when the application is locally installed on each user's machine. \parencite{jensen2017}\paragraph{}
However, these benefits of desktop applications come with a significant drawback.
Developing desktop applications requires developers to be proficient at languages like C++, Objective-C or C\#. 
For a portion of developers, this can be a significant barrier to entry because it means learning an all-new language and in some cases even frameworks as well.
