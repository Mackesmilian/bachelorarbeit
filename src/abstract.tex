Web applications have been gaining in popularity and complexity
over the recent years and have resulted in a de-facto move from 
the desktop to the web browser. 
More internet users, faster connection speeds, and other reasons
such as financial ones have all contributed to this rise in 
popularity of web apps. 
However one can ask themselves if this really is the only way 
forward for bringin applications to users. 
After all, relying on a constant internet connection is not 
always a guarantee, for instance.
To overcome this issue several frameworks have been developed 
which claim to overcome this hurdle by allowing web developers 
to use their expertise and experience to create desktop 
applications. 
One of such solutions is Electron, a framework promising to 
enable creating native desktop applications using only web 
technologies such as JavaScript, HTML, and CSS. 
The goal of this thesis is therefore to evaluate how Electron
achieves this and what advantages and disadvantages developers 
need to take into account.
To find conclusive answers to this issue a case study was conducted which shows the 
process of developing such an application from start to finish. 
Resulting from this case study considerations were defined which developers 
and organisations alike may take into account when deciding upon 
using Electron. 
This has also raised relevant questions to be answered in the future
such clearly quantifying how much extra work is needed for Electron 
and how Electron works in a productive environment in regards to 
maintenance, versioning and further development. 