\externaldocument{main.tex}
As mentioned before the intention of the application is to facilitate efficient time keeping 
for employees of Comm-Unity EDV GmbH. 
As with any business-oriented time keeping program it is central to be able to book 
time worked on specific data points not only to be able to bill customers correctly but also to 
create a clearer picture of which project and/or customer require what amount of attention by 
employees.
In this specific proof-of-concept said data points are limited to customers and projects.\paragraph{}
These can be easily extended and customised depending on domain-specific requirements and needs but as far 
as this example goes, a generalised approach is sufficient and also required as to extrapolate results 
on other use cases.\paragraph{}
The below described requirements have been deducted from decades long experience within the company in question.
As a comparable application has been in use for 20 years, users of this tool and engineers who were involved
in designing the legacy application have an extensive knowledge of what improvements were required.
These improvements were then defined as new features and continuously re-evaluated throughout development.
These re-evaluation cycles used input from software engineers with extensive company specific knowledge and
experienced users.
Therefore the requirements for the app created and discussed within this thesis were adopted from the existing
application developed for Comm-Unity EDV GmbH.\paragraph{}
The application will be structured into four different views. 
The main view shows a list of entries which each represent a data point.
Said entry has a start and end time and a project and customer are assigned to each entry. 
As the number of customers and projects increases over time it becomes increasingly difficult for 
employees to quickly find the correct values to attribute their entries to.
To make this easier to use users can create templates which limit the possible data points one can
choose.\paragraph{}
The second view is the so called template view which shows a list of the aforementioned templates.
Users can create, update and delete templates which each have a unique ID, a name and a list of projects 
and customers.\paragraph{}
The third view is an overview of customers, which can be created, updated and removed. 
Each customer is comprised of a name and an address.\paragraph{}
The fourth view is similar to the customers view but represents an overview of projects, which 
can also be created, updated and removed.
Each project contains a name and whether said project is active or not.\paragraph{}
