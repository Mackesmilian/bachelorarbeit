Now the next obvious question is why a framework such as Electron is needed at all.
After all, it is "just" a way to have desktop applications developed as web applications.
So why not just develop native desktop applications or traditional web applications depending on the use case?
To answer this one has to examine the bigger picture:\par
Over the past decade it seems as though software pricing has moved from perpetual licenses towards subscription-based
models.
If one examines the data regarding end-user spending on cloud applications it is clear that the
\emph{Software as a Service} (SaaS) model has grown considerably in revenue and is projected to do so in the future:
The worldwide end user spending for Software as a Service has increased from 31.4 billion US Dollars in 2015 to 120
billion US Dollars in 2020.
It is projected this growth will progress with spending reaching 171.9 billion dollars in 2022.\cite{gartner2021}\par
Furthermore, Gartner (2021) forecasts that by 2026, cloud spending will exceed 45\% of all enterprise IT spending, up from
17\% in 2021.
This impressive growth can be attributed to two reasons.
Reasons either technical and/or financial in nature.
One financial benefit of SaaS is economies of scale:
By hosting the application centrally and by extension aggregating users together, providers can benefit financially from
leveraging economies of scale.
At the simple end, this means benefiting from volume pricing on hardware such as data centers, servers, space and so on.
Taking this idea further, SaaS providers can also cut costs by sharing hardware across their customers.
It is not cost-effective to use one machine for each customer, instead resources should be shared and dynamically
allocated on-demand to each customer's needs.
Similarly, as user count increases, the cost of adding on single user decreases.
These and other reasons are a big financial motivator for providers of software to switch to the SaaS model.\par
However, technological reasons play a large role as well.
According to Jacobs (2005) the ever-increasing maturity of the Web is a major contributor for the rise in popularity of
SaaS\@.
Browsers are significantly more powerful than ever, internet access is more widespread and faster, and the number
of robust frameworks for web development (be it front-end or back-end), make creating a complex web application easier
than ever before.\cite{akamai2017, statista2021}
Additionally, SaaS can be developed more easily than native desktop applications because in a web environment there is no
need to port to all the different platforms one wants to support. 
Moreover the software provider has strict control over when updates occur in contrast to users having to manually update their 
desktop applications and should bugs occur, developers can reproduce said bugs much more easily with real-world data.\cite{jacobs2005}\par
However, web applications come with their disadvantages. Due to browser security access to a user's file system is not possible, for example.
Furthermore, some users may have a personal preference towards traditional desktop applications. 
