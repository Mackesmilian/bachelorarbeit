\externaldocument{main.tex}
Naturally, in a rapidly evolving environment which software engineering and even more so 
web development is, developers have multiple approaches to solving the same issue, which 
in this case would be the development of a web application or even desktop application.
Having outlined why web applications have experienced such a rise in popularity and how
Electron presents a novel concept of overcoming web application's inherent shortcomings, 
one question arises in the context of software engineering and development:
What advantages and shortcomings does Electron present over traditional web applications? \paragraph{}
As relevant literature and resources are not available to answer this question a case study
will be conducted to answer this research question. 
The research question encompasses other questions with more specific objectives. 
These questions are: 
Which significant shortcoming and advantages exist for developers?
Which significant shortcoming and advantages exist for users?
Where can the advantages of each be best leveraged in a commercial environment?
The case study will follow the design priniciples laid out by \textcite{caseStudy}. \paragraph{}
The case study consists of an application of four different view with similar albeit different tasks. 
The purpose of this application is to serve as a time keeping tool for employees of a company.
The time worked can be recorded depending on multiple factors such as customers, projects and other
internal, domain-specific aspects.\paragraph{}
The following sections will focus on the case study and more specifically 
development of the Electron application after which conclusions
will be made in order to answer the previously posed research questions.
The conclusion will be based on qualitative data collected during
the case implementation and evaluation of that data will be made in 
accordance with criteria defined by relevant literature such as \textcite{frameworksEfficiency} 
and \textcite{backEndComparison} which describe criteria for frameworks to be
evaluated against.\paragraph{}