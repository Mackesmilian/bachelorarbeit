\externaldocument{main.tex}
After describing the case this chapter will now focus on answering the research questions.
As the case implementation has shown Electron is a very flexible framework 
which allows web developers to create a desktop application with very little 
additional knowledge required. 
This is arguably one of the most significant advantages of Electron which is 
also supported by the project health criterion outlined by \textcite{ratingsFW} 
which states that metrics such as the number stackoverflow.com tags correlate 
with how well known and mature a framework is. \parencite{backEndComparison}
Electron for instance has - as of April 2022 - 13525 questions tagged whereas
other competing framworks such as the briefly described NW.js is only tagged 
in 307 questions.\paragraph{}
According to \textcite{frameworksEfficiency} other evaluation criteria include
time spent developing pages, time spent modifying the existing application and size
of written code. 
As measurements in these cases have not been made, quantitative data cannot 
be provided. 
However as this case study highlights Electron in comparison to web applications
other conclusions can be drawn.
As the case study has shown the vast majority of developed code is euqivalent to 
developing a web application with a traditional client - server architecture. 
This means that Electron specific code was limited and therefore developers can 
safely assume that using Electron will not significantly increase the size of 
their written code and therefore time spent developing, debugging, testing, and
deploying. \paragraph{}
However, Electron is not without fault as has been shown by this case study.
One disadvantage is the integration of Angular into Electron has not worked flawlessly 
with some issues such as breaking changes in different versions and hard to 
degug behaviour having somewhat slowed down development. 
Another drawback of Electron which was not discussed within this case study 
is the question of security. 
As claimed by \textcite{frameworksSecurity} Electron applications often display 
security issues and vulernabilities such as cross-site-scripting or remote code execution.
Even popular and widely used applications such as Discord have had significant 
security issues in the past. \parencite{discordVulernabilities}\paragraph{}
Moving on to the benefits for users rather than developers.
As far as users go, Electron does not exhibit any disadvantages. 
For the average user it is frankly of no importance whether their application
runs in a browser or on desktop as long as their needs and wants are met which
is an area where Electron can show its strengths. 
This case study has shown developers have at least the same degree of 
design freedom in regards to user experience and user interface design 
with added benefits such as offline functionality and others as described
during this case study. 
This freedom for creativity means that developers can design their applications
in a way their users are accostumed to while also being able to create
interfaces following commonly accepted usability rules such as those
laid out by \textcite{usabilityHeuristics}.\paragraph{}
Lastly, the question of how this technology can best be used in 
a commercial environment is left to answer.
This depends on many different factors with the choice 
ultimately being at the organisation's and its 
developers' discrection. 
However, one can create certain criteria where Electron would at the 
very least be worthy of a consideration. 
As this case study has shown the only required knowledge is that 
of web development. 
That is to say that if an organisation has sufficient knowledge and 
experience in creating web applications, using Electron will require 
little if any additional training or learning. 
Another point to consider is the case which the application should serve.
Questions to ask oneself can include whether the featureset of Electron 
can be fully utilised and if Electron solves any identified problems. 
One can for example consider a simple \acrfull{crud} application within 
a business context with the purpose of managing some data. 
Electron would in this case most likely not add any additonal value 
as such basic functions can be implemented without any drawbacks with 
a classic client-server application or even other frameworks potentially better
suited for this purpose.\paragraph{}
If however offline functionality is important, the access to a client's file 
system is beneficial or even if a specific user base benefits from not having
to update their browser then Electron is at the very least worth considering.\paragraph{}
As this case study and by extension thesis can only highlight certain limited 
aspects of Electron numerous considerations for future research can be made. 
For instance an analysis of how Electron affects development time and complexity 
can be made and then analised and compared to other approaches be it desktop
or web development.
Another consideration would be how Electron fits into a productive and 
commercial environment in regards to versioning, maintenance, and further development 
which focus on any identified problem areas which could result in increased cost. 
