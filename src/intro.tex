In December 2012 software engineer Cheng Zhao joined GitHub's team, having previously worked for Intel developing
node-webkit, with the task of porting the Atom editor from using Chromium Embedded Framework to node-webkit.
Node-webkit being a Node.js module developed by Roger Wang which combined the browser engine used by Chromium - WebKit - 
with Node.js, making Node.js modules accessible from JavaScript code running inside a web page. \parencite{jensen2017}\paragraph{}

Porting to node-webkit proved difficult, so GitHub abandoned that approach, and it was decided that a new native shell
for Atom would be created.
Said shell was dubbed \emph{Atom Shell} and after development was finished and the Atom editor was open sourced
by GitHub, Atom Shell soon followed suit and was renamed to \emph{Electron}.
Initially developed as a way to deliver an editor, numerous widely known applications like Slack, Discord and Visual
Studio have started using Electron to develop and deliver their desktop applications.\parencite{electronDocs}
But what exactly is Electron?\paragraph{}