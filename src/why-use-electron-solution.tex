\externaldocument{main.tex}
However, web applications do come with disadvantages. 
As described in \ref{subsubsec:desktop-applications} web apps have their shortcomings such as 
browser security preventing access to a user's file system or having no access as soon as the internet connection
fails.\paragraph{}
This is where frameworks such as Electron manage to strike the near-perfect balance between desktop application and web app.
For instance the drawback of not having access to a user's PC's file system does not apply to applications 
developed using Electron, as the npm module \emph{osenv} can for example retrieve the user's home folder among 
other environment settings. \parencite{osenv}\paragraph{}
Additionally, the disadvantage of having to consider different browsers (and versions thereof) are a non-issue
with electron because Electron uses Chromium as outlined in \ref{subsec:what-is-electron}. 
Furthermore, internet access is not a requirement with Electron which means applications can have some offline
functionality as opposed to web apps.\par
These are some features and advantages of Electron, though not an exhaustive list. \parencite{electronDocs}\paragraph{}
Ultimately, it is at the developer's discretion which form of software to use.
Native desktop applications, web applications or applications developed using frameworks such as Electron 
all have advantages and disadvantages and it is important to consider
which solution fits an application's and/or user's needs best.\paragraph{}
During the course of this bachelor thesis, Electron will be evaluated as a framework for developing desktop applications
by implementing an example project. 
The aforementioned considerations and points will be examined in development of said application and 
finally an evaluation will be made on how effective of an alternative to web applications Electron is 
and whether the biggest shortfalls of web applications can be eliminated or mitigated by 
using Electron.